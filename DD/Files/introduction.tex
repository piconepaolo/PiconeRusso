\setlength{\headheight}{13.59999pt}
\addtolength{\topmargin}{-1.59999pt}

\subsection{Purpose}
The motivation behind the existence of the CodeKataBattle platform is to provide students with a dedicated platform for practicing their programming skills. The platform aims to create a competitive environment where teams of students can participate in programming tournaments and solve programming challenges.

By offering a platform specifically designed for programming practice, CodeKataBattle aims to provide students with a structured and engaging way to improve their coding abilities. The competitive nature of the platform adds an extra layer of motivation and excitement, encouraging students to push their limits and strive for excellence.

Overall, the motivations behind the existence of the CodeKataBattle platform are to create a dedicated space for programming practice, foster healthy competition among students, and provide a means for tracking and improving programming skills.

\subsection{Scope}
As mentioned above in the purpose section, the CodeKataBattle platform aims to provide students with a dedicated space for programming practice. The architectural style of the platform will be a three-tier architecture. The web interface is part of the first layer of the three-tier architecture. The three-tier architecture is a client-server software architecture pattern that separates the user interface, application processing, and data management functions into separate modules. The students and the educators will interact with the presentation layer, in particular with the Web App. Part of the presentation layer is also the Web Server that will handle the HTTP requests and responses. The second layer is the application layer, represented by the Application Server, which is responsible for the business logic of the application. The third layer is the data layer, represented by the Database Server, which is responsible for storing and retrieving data.

\subsection{Definitions, Acronyms, Abbreviations}
\subsubsection{Definitions}
\begin{itemize}
    \item \textbf{Educator}: a person who is responsible for creating and managing tournaments and battles
    \item \textbf{Student}: a person who is participating in tournaments and battles
    \item \textbf{Team}: a group of students that participate in a battle
    \item \textbf{Battle}: a programming challenge that takes place in a tournament
    \item \textbf{Code Kata}: a programming challenge that defines a battle
    \item \textbf{GitHub}: a web-based hosting service for version control using Git
    \item \textbf{GitHub Actions}: a feature of GitHub that allows automating tasks
    \item \textbf{GitHub repository}: a storage space where the project files are stored
    \item \textbf{GitHub fork}: a copy of a repository
    \item \textbf{Platform}: The interface that allows the interaction between the user and the system
    \item \textbf{System}: The software that implements the functionalities of the platform
\end{itemize}
\subsubsection{Acronyms}
\begin{itemize}
    \item \textbf{CKB}: CodeKataBattle
\end{itemize}
\subsubsection{Abbreviations}
\begin{itemize}
    \item \textbf{R}: Requirement
    \item \textbf{API}: Application Programming Interface
\end{itemize}
\subsection{Revision history}

\subsection{Reference documents}
This document is based on:
\begin{itemize}
    \item The specification of the DD assignment of the Software Engineering 2 course
    \item The slides of the Software Engineering 2 course
\end{itemize}


\subsection{Overview}
This document is structured as follows:
\begin{itemize}
    \item \textbf{Introduction}: this section provides a general overview over the product and the decision made about the adoption of some architecturel styles. It also contains the definitions, acronyms and abbreviations used in the document.
    \item \textbf{Architectual design}: this section contains the description of the architectural design of the system. It includes the high level diagram of the components and how they interact with each other. It also contains the description of the components and the interfaces they expose. A section of this chapter is dedicated to the deployment diagram of the system. Finally, this chapter contains the description of the runtime view of the system, which includes the sequence diagrams of the main functionalities of the system.
    \item \textbf{User Interface Design}: this section contains the mockups of the user interface of the system. It has the same assets as the RASD document.
    \item \textbf{Requirements Traceability}: this section contains the mapping between the requirements defined in the RASD document and the components of the system defined in the component diagram.
    \item \textbf{Implementation, Integration and Test Plan}: this section contains the description of the strategy adopted to implement and integrate the components of the system. It also contains the description of the testing strategy adopted.
\end{itemize}
