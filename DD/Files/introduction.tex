\setlength{\headheight}{13.59999pt}
\addtolength{\topmargin}{-1.59999pt}

\subsection{Purpose}
The motivation behind the existence of the CodeKataBattle platform is to provide students with a dedicated platform for practicing their programming skills. The platform aims to create a competitive environment where teams of students can participate in programming tournaments and solve programming challenges.

By offering a platform specifically designed for programming practice, CodeKataBattle aims to provide students with a structured and engaging way to improve their coding abilities. The competitive nature of the platform adds an extra layer of motivation and excitement, encouraging students to push their limits and strive for excellence.

Overall, the motivations behind the existence of the CodeKataBattle platform are to create a dedicated space for programming practice, foster healthy competition among students, and provide a means for tracking and improving programming skills.

\subsection{Scope}
...

\subsection{Definitions, Acronyms, Abbreviations}
\subsubsection{Definitions}
\begin{itemize}
    \item \textbf{Educator}: a person who is responsible for creating and managing tournaments and battles
    \item \textbf{Student}: a person who is participating in tournaments and battles
    \item \textbf{Team}: a group of students that participate in a battle
    \item \textbf{Battle}: a programming challenge that takes place in a tournament
    \item \textbf{Code Kata}: a programming challenge that defines a battle
    \item \textbf{GitHub}: a web-based hosting service for version control using Git
    \item \textbf{GitHub Actions}: a feature of GitHub that allows automating tasks
    \item \textbf{GitHub repository}: a storage space where the project files are stored
    \item \textbf{GitHub fork}: a copy of a repository
    \item \textbf{Platform}: The interface that allows the interaction between the user and the system
    \item \textbf{System}: The software that implements the functionalities of the platform
\end{itemize}
\subsubsection{Acronyms}
\begin{itemize}
    \item \textbf{CKB}: CodeKataBattle
\end{itemize}
\subsubsection{Abbreviations}
\begin{itemize}
    \item \textbf{WP}: World Phenomena
    \item \textbf{SP}: Shared Phenomena
    \item \textbf{G}: Goal
    \item \textbf{R}: Requirement
    \item \textbf{UC}: Use Case
    \item \textbf{UI}: User Interface
    \item \textbf{API}: Application Programming Interface
\end{itemize}
\subsection{Revision history}

\subsection{Reference documents}
This document is based on:
\begin{itemize}
    \item The specification of the RASD assignment of the Software Engineering 2 course
    \item The slides of the Software Engineering 2 course
\end{itemize}

\subsection{Document structure}
This document is structured as follows:
\begin{itemize}
    \item \textbf{Introduction}: it provides a general description of the product, its purpose and the goals that the project aims to achieve. It also contains the scope of the product, the phenomena that the product will interact with and the shared phenomena between the product and the world. Finally, it contains the definitions, acronyms and abbreviations used in the document.
    \item \textbf{Overall Description}: it's a high level description of the product. It contains the product perspective, the product functions, the user characteristics, the constraints, the assumptions and the dependencies of the product.
    \item \textbf{Specific Requirements}: it contains the functional and non-functional requirements of the product. It also contains the use cases, the sequence diagrams of the most important use cases and the external interfaces.
    \item \textbf{Formal Analysis}: it contains the Alloy model of the product.
    \item \textbf{Effort Spent}: it contains the number of hours spent by each member of the group to redact this document.
    \item \textbf{References}: it contains the list of the documents used to redact this document.
\end{itemize}
