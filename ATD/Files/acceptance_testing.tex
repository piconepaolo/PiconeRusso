The tests performed are based on the functions implemented in the code.
The following list shows the tests grouped by use case:
\begin{itemize}
    \item \textbf{Educator and Student Login}: the test case was performed by using the provided front-end interface. The login was performed by logging in with a GitHub account. There were successfully created and logged in both an Educator and a Student account.
    \item \textbf{Educator and Student Registration}: the test was performed by using the provided front-end interface by signing up with a GitHub account and then selecting the role for the newly created user: Educator or Student. The registration was successful even if there is no control on whether the user is actually an educator or a student.
    \item \textbf{Tournament and Battle Creation}: the test was performed by using the provided front-end interface and by using an Educator account. The creation of both a Tournament and a Battle was successful.
    \item \textbf{Subscribe to a Tournament or a Battle}: After the creation of a Tournament and a Battle, the test was performed by using the provided front-end interface and by using a Student account. The subscription to both a Tournament and a Battle was successful.
    \item \textbf{Subscribe to a Tournament after the registration deadline}: The test was performed by creating a Tournament and setting an arbitrary registration deadline. Once done that, the entry related to the registration deadline for the Tournament was modified manually in the database. After permforming the update, the test was performed by using the provided front-end interface and by using a Student account. The subscription to the Tournament was successful even if the registration deadline was already passed. This is a bug that should be fixed since it does not respect the requirements of the project. The test was not successful.
    \item \textbf{Notification on newly Created Tournament or Battle}: The test was performed by creating a Tournament and a Battle and by using the provided front-end interface by using an Educator account. The notification was successfully received by the Student account in the corresponding notification page.
    \item \textbf{End of a Tournament or a Battle}: The test was performed by creating a Tournament and a Battle and by using the provided front-end interface by using an Educator account. Once in the Tournament page or in the Battle page, the end of the Tournament or the Battle was manually triggered using the corresponding button. The end of the Tournament or the Battle was successfully received by the Student account in the corresponding notification page.
    \item \textbf{Code Push}: The test was performed by first creating a Tournament and a Battle. Then by using a student account the subscription to the Tournament and the Battle was performed. Once set up the student repository with the appropriate workflow (provided in the ITD code), the student pushed the code to the repository. The push was successful and the request was correctly received by the platform.
\end{itemize}