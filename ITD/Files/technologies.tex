In the current section are described the programming languages, frameworks and technologies used to develop the prototype of the CKB platform.

\subsection{Programming Languages and Frameworks}
The programming language used to develop the prototype of the CKB platform is Python and the api are built with FastAPI.
The choice of Python is due to the familiarity of the team with the language. The Frameworks FastAPI was chosen since it brings several advanteges to the development of the api, such as the automatic generation of the OpenAPI documentation, its performace and its ease of use and learning. Some disadvantages of FastAPI are the lack of some features that are present in other frameworks, such as the lack of a built-in ORM and the lack of a built-in authentication system. It also offers a lot of flexibility, which can be both an advantage and a disadvantage depending on the context it's used in.

\subsectionmark{Technologies}
The database used is MongoDB, a NoSQL database. The motivation behind the choice of MongoDB is because, like mentioned in the DD document, a document-based database is a good fit for the project and MongoDB is one the possible choices for this type of DB. MongoDB brings some advanteges such as a robust library (\textbf{PyMongo}) in Python to interact with, since the Python data stractures can be easily converted in document to insert and retrieve from the database. One of the main disadvantages of MongoDB is the difficulty to perform complex queries and the lack of transactions, which can be a problem in some contexts.
Another technology used is Docker, which is used to containerize the application and the database. The choice of Docker was to make the development among the team members more consistent and to make the deployment of the application easier. Docker also brings some advantages such as the possibility to run the application in different environments without the need to install the dependencies and the possibility to run the application in a cloud environment. The main disadvantage of Docker is the overhead that it brings to the application, since the application is running in a container, which has to be configured and managed properly.

\subsection{Main libraries}
One of the main libraries used in this project is Pydantic. Pydantic is used to validate the data that is sent to the api and to convert the data to the correct type. It offers a lot of methods to validate data received by the api. The choice of Pydantic was due to the fact the has a good documentation and it's very powerful when used by FastAPI. An example could be the generation of the openapi.json file, which is done automatically by FastAPI using Pydantic. \\
Another important mention is PyTest to perform tests on the api. PyTest is a very powerful library to perform tests in Python and it's very easy to use and to learn. It offers a lot of features to perform tests such as fixture and mocks. It is also very easy to use and run the tests.