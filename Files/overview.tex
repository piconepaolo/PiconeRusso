\subsection{Product Perspective}
\subsubsection{Scenarios}
\begin{enumerate}
    \item \textbf{Educator creates a battle for a tournament} \\
    The educator who wants to create a battle in a tournament for which he is the creator, or for which he has the permission to create battles.
    The creation of a battle consists of in:
    \begin{itemize}
        \item Set the deadline for registration of teams
        \item Set the deadline for the submission of solutions
        \item Set the minimum and maximum number of students per team
        \item Upload the code kata to the CKB platform
    \end{itemize}
    Once the battle is created, the students registered at that tournament are notified by CKB about the upcoming battle. The CKB platform will also create a new repository on Github for the battle, and will invite the students to fork it.
    \item \textbf{Student invite other students to join the team} \\
    Roberto is a student who wants to participate in a battle. He has already registered in the tournament in which the battle has been created, and he has been notified about the upcoming battle. He wants to invite other students to join his team for the battle. He can do this by sending an invitation to the other students, who will receive a notification about the invitation. The other students can accept or decline the invitation. If they accept, they will be added to the team and they will be able to compete as a team in a battle.
    Even if the students invited have accepted the invitation the partecipation to the battle is not guaranteed since the educator could have set a maximum number of students per team.
    \item \textbf{Student push the solution of the battle to the forked repository} \\
    The team A have been working on the solution for the battle for a while and it seems to be passing all the tests provided in the code kata. Since the time being is before the submission deadline the team push the solution to the forked repository. This will trigger the CKB platform to evaluate the solution and update the score of the team in the battle. In the case that the submission date is after the submission deadline the push will not trigger another evaluation of the solution by the CKB platform.
    \item \textbf{Educator manually updates the score of a team} \\
    Giacomo is an educator who has created a battle for a tournament in which he has been invited by another collegue. Once the submission deadline is expired he wants to manually assess the score of a team to cover all the aspects that the CKB platform cannot evaluate. To do this he access through the CKB platform the battle page and he can see the list of teams that have submitted a solution. He can then manually update the score of a team by clicking on the `Update score' button. This will open a form where he can modify the current score of the team previously evaluated by the platform. Once he is satisfied with the score he can click on the `Save' button to save the new score for the team. The new score will be visible to the team and to the other students subscribed to the tournament.
    \item \textbf{CKB notifies students about the result of the tournament} \\
    When an Educator closes a tournament, the CKB platform will notify all the students subscribed to the tournament about the result of the tournament. The notification will contain the list of the teams that have participated in the tournament and their final score.

\end{enumerate}
\subsubsection{Domain class Diagram}
In \figurename~\ref{fig:domain_class_diagram} is shown the domain class diagram of the CKB platform. This diagram shows the main entities of the system and their relationships.
\begin{itemize}
    \item \textbf{User}: is the main entity of the system. It represents a user of the CKB platform. It can be either a student or an educator.
    \item \textbf{Student}: is a user of the CKB platform. It can participate in tournaments and battles. It can also create teams and invite other students to join them.
    \item \textbf{Educator}: is a user of the CKB platform. It can create and manage tournaments and battles. He can also invite other educators to join a tournament. He also uploads the code kata to the CKB platform for a battle.
    \item \textbf{Tournament}: is a competition between teams of students. It is created by an educator and it can be joined by students. It can contain multiple battles.
    \item \textbf{Battle}: is a competition between teams of students. It is created by an educator and it can be joined by students. It is part of a tournament and it can contain multiple teams.
    \item \textbf{Team}: is a group of students that compete together in a battle. It is created by a student and it can be joined by other students.
    \item \textbf{Invitation}: is a request sent by a student to another student to join a team. It can be accepted or declined by the receiver.
    \item \textbf{Code Kata}: is the description of the problem that the student have to solve in a battle. It contains the description of the software project to be implemented, the test cases that the solution must pass and the build automation scripts.
    \item \textbf{Submission}: is the code that a team of students have implemented for a battle or a subsequent version of a previous submission. It is pushed to the forked repository of the battle by a team member. The CKB platform will evaluate the submission and update the score of the team.
\end{itemize}

\begin{figure}[H]
    \centering
    \includegraphics[width=1\textwidth]{Diagrams/DomainClassDiagram.jpg}
    \caption{Domain class diagram}
    \label{fig:domain_class_diagram}
\end{figure}

\subsubsection{Statecharts}

\textbf{Statechart of a Tournament} \\
The statechart in \figurename~\ref{fig:statechart_tournament} shows the possible states of a tournament. A tounament can be in two principal states: ``In Progress'' or ``Closed''. A tournament is in the ``In Progress'' state when it is created by an educator. When the educator closes the tournament, it will be in the ``Closed'' state.
\begin{figure}[H]
    \centering
    \includegraphics[width=1\textwidth]{Diagrams/TournamentStateChart.jpg}
    \caption{Statechart of a Tournament}
    \label{fig:statechart_tournament}
\end{figure}
\textbf{Statechart of a Battle} \\
The statechart in \figurename~\ref{fig:statechart_battle} shows the possible states of a battle. When the educator creates a battle it will be in initial ``Created'' state. When the date of the \textit{Start date} is passed the battle will be in the ``In Progress'' state. In this state the students can subscribe to the battle and the teams can submit their own solutions. When the \textit{Submission deadline} is passed the battle will be in the ``Consolidation'' state. In this state the educator can manually update the score of the teams. When the educator closes the tournament, it will be in the ``Closed'' state.
\begin{figure}[H]
    \centering
    \includegraphics[width=1\textwidth]{Diagrams/BattleStateChart.jpg}
    \caption{Statechart of a Battle}
    \label{fig:statechart_battle}
\end{figure}

\subsection{Product Functions}
\subsubsection{User registration and login}
The CKB platform allows the users to register and login to the platform. The registration is required to access the platform and to use its functionalities. The registration process requires the user to provide a valid email address and a password. The email address will be used to identify the user in the platform \ie{User or Educator}. The password will be used to authenticate the user when he wants to login to the platform. The CKB platform will send a confirmation email to the user to verify the email address. The user will be able to login to the platform only after the email address has been verified.
\subsubsection{Creation of a tournament}
The CKB platform allows the educators to create tournaments. The Educator will be prompted to provide the name of the tournament and the deadline for the registration. When the tournament is created, the CKB platform will send a notification to all the students registered to the platform. The Educator, that has done the creation, will be able to invite other educators to join the tournament.
\subsubsection{Creation of a battle}
The CKB platform allows the educators to create battles. The Educator will be prompted to provide the name of the battle, the deadline for the registration, the deadline for the submission of the solutions, the minimum and maximum number of students per team and the code kata. When the battle is created, the CKB platform will send a notification to all the students subscribed to the tournament. The CKB platform will also create a new repository on Github for the battle, and will invite the students to fork it.
\subsubsection{Creation of a team}
\subsubsection{Invitation to join a team}
\subsubsection{Submission of a solution}
\subsubsection{Evaluation of a solution}
\subsubsection{Manual update of the score}
\subsubsection{Notification of the result of a tournament}
\subsubsection{Creation of the repository to sumbit the solution}
